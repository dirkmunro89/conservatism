\documentclass[12pt]{article}
\usepackage{graphicx}
\usepackage{hyperref}
\usepackage[landscape, margin=2cm]{geometry}
\usepackage{multicol}
\usepackage{pgffor}
\usepackage{verbatim}
\usepackage{amsmath,bm}
\usepackage{mathrsfs}
\usepackage{fancyhdr}
\pagestyle{fancy}
\renewcommand{\headrulewidth}{0pt}
\usepackage{sectsty}
\sectionfont{\fontsize{12}{15}\selectfont}
%
\newcommand{\f}{\textrm{F}}
\newcommand{\g}{\textrm{G}}
\newcommand{\x}{{x}}
\renewcommand{\d}{\textrm{d}}
%
%
%
\begin{document}
\Huge
%
\centering
\vspace*{20mm} 
\textbf{Groenwolds's relaxed conservatism}

\bigskip

\small
\raggedright

See algorithm in appended Python code. The objective and the constraint function is given verbatim below. (Notice the naive choice of (a quadratic approximation to) a function linearised in terms of reciprocal intervening variables, \verb{dd{$\ldots$. The reader is encouraged to, for example, uncomment the analytic second order information.)

\footnotesize
\verbatiminput{fun.py}

\thispagestyle{empty}

\newpage

\pagenumbering{arabic} 
\renewcommand{\thepage}{Iteration \arabic{page}}
%
\foreach \k  in {0,...,100}{
\IfFileExists{itr_\k.eps}{
%
\begin{minipage}[t]{8cm}
\normalsize
\vspace{-10cm}
\input{tmp2_\k.tex}
\end{minipage} 
\begin{minipage}[t]{14cm}
%\begin{figure}[htb!]
\centering
\includegraphics[width=14cm]{itr_\k.eps}
%\end{figure}
\normalsize
\end{minipage}
\normalsize
\raggedright
%

\input{tmp1_\k.tex}
%
\newpage
\large
}
}{}
%
\small
%\begin{multicols}{2}
%\verbatiminput{slides.tex}
%\end{multicols}
%
%\newpage
%\begin{multicols}{2}
\vspace*{20mm} 
\Large
\thispagestyle{empty}
\href{https://raw.githubusercontent.com/dirkmunro89/conservatism/main/groen.py}{Python code}
%\end{multicols}
%
\end{document}
